\documentclass{hw_grad}

% === RECIPES =============================================================================================================================

% ~~ FIGURE ~~
%\begin{figure}[h]
%	\centering
%	\includegraphics[scale=0.45]{imaginativeAndGenerative}
%	\caption{Relationship between Generative and Imaginative}
%	\label{imaginativeAndGenerative}
%\end{figure}

% ~~ Answer Boxes ~~
% Inside Eq: \boxed{}
% Text:      \fbox{}
% Region:    \begin{framed} ... \end{framed}

% ~~ Tables ~~
%\begin{tabular}{ l c r }
%	1 & 2 & 3 \\
%	4 & 5 & 6 \\
%	7 & 8 & 9 \\
%\end{tabular}

% ~~ Code Listings ~~
%\begin{lstlisting}[language=c] 
%  // Code here
%\end{lstlisting}

% ___ END RECIPES _________________________________________________________________________________________________________________________


\title{HW\#\#: ASSIGNMENT TITLE}
\duedate{Due: YYYY-MM-DD}
\class{CSCI-5302: Advanced Robotics, 2019 Spring}
\institute{University of Colorado Boulder}
\author{James Watson, 105754866}

\begin{document}
	\maketitle
	
	
	\begin{thebibliography}{1}
		
		\bibitem{label1} Kuffner, James J., and Steven M. LaValle. ``RRT-connect: An efficient approach to single-query path planning.'' In Robotics and Automation, 2000. Proceedings. ICRA'00. IEEE International Conference on, vol. 2, pp. 995-1001. IEEE, 2000.
		
%		\bibitem{Erdos01} P. Erd\H os, \emph{A selection of problems and
%			results in combinatorics}, Recent trends in combinatorics (Matrahaza,
%		1995), Cambridge Univ. Press, Cambridge, 2001, pp. 1--6.

%		\bibitem{ConcreteMath}
%		R.L. Graham, D.E. Knuth, and O. Patashnik, \emph{Concrete
%			mathematics}, Addison-Wesley, Reading, MA, 1989.

%		\bibitem{Knuth92} D.E. Knuth, \emph{Two notes on notation}, Amer.
%		Math. Monthly \textbf{99} (1992), 403--422.

%		\bibitem{Simpson} H. Simpson, \emph{Proof of the Riemann
%			Hypothesis},  preprint (2003), available at 
%		\url{http://www.math.drofnats.edu/riemann.ps}.
		
	\end{thebibliography}
	
\end{document}