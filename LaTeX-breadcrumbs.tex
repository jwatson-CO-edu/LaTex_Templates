% == tabular Formatting ==

\begin{tabular}[t]{c|c|c|lc}
% [t] : {t,c,b} : align the surrounding line to the top, center, or bottom of table if it is inline
% {l|r|cccc}
%    - One letter per column
%    - Each letter {l,c,r} represents left, center, right alignment
%    - Pipes | between columns will draw a vertical line between those columns
%    - \hline at the end of a line draws a horizontal line under that column
%    - \cline{N-M} is similar to \hline but only spans some of the columns

% Define a new macro to put bottom padding for cell, use '\B'
\newcommand\B{\rule[-2ex]{0pt}{0pt}} 

% Stretch out the rows of just this table
{\renewcommand{\arraystretch}{1.9}
\begin{tabular} ...
\end{tabular}
} % You have to close 'renewcommand'

\cline{1-4} % A horizontal line across columns 1 through 4 only

% == End Tabular ==

% Create a figure space for specialized formatting and place it inline with text
\begin{figure}[h] % place the figure in sequence with text [here]
\centering % center the figure on the page, independent of surrounding formatting
\end{figure}

% multiple lines within a table cell
\vtop{\hbox{\strut $ Q(1,1) = \dfrac{(93+95+94)}{3.0} $ }\hbox{ \strut $=  94 $ }} &

% insert some horizontal space in a line
\hspace{5 em} % 'em' or any other recognized unit

$\ $ \\ % force an "empty" line

\newpage % page break

% Elements of an inline equation are in the big style
$\displaystyle <EQUATION> $

% Put a box around your answer
$\displaystyle EQUATION = \boxed{ANS} $
\boxed{\text{False}} % Also works outside of equation, have to use, but contents are equation style
                     %unless you negate it with '\text{}'

% Put a box around a bunch of stuff
\fbox{ <A BUNCH OF STUFF> }

% Two tables side by side, insert a spacing to override the default newline
\begin{tabular} ... \end{tabular}
\quad % The spacing that I choose over the default separation
\begin{tabular} ... \end{tabular}

\texttt{ monospace text } % monospace text inline with regular text

% Checklists , To-Do Lists
\usepackage{wasysym}
$\Square$ , $\CheckedBox$ , $\XBox$

% == Color Formatting ==

\usepackage{xcolor}
% Prefined: red, green, blue, cyan, magenta, yellow, black, gray, white, darkgray, lightgray, brown, lime, olive, orange, 
%           pink, purple, teal, violet  ~~~~ Or define your own color names:
\definecolor{darkgreen}{rgb}{ 0.28 , 0.6 , 0.28 } % Must be expressed in decimal fractions
\textcolor{darkgreen}{YOUR TEXT HERE}

% == End Color ==

\begin{minipage}[t]{0.27\textwidth}
  % Put stuff in its own little environment, like a column
\end{minipage}

% Print a number with comma thousands separators but without an annoying space
$32,\!154,\!192$

Spaces in mathematical mode.
% Spaces are arranges smallest to largest
$$ 
\begin{align*}
f(x) =& x^2\! +3x\! +2 \\ % Negative space, consumes a tiny pit of space in front of it
f(x) =& x^2+3x+2 \\       % No space or automatic space for context-sensitive spacing
f(x) =& x^2\, +3x\, +2 \\ % Tiny space
f(x) =& x^2\: +3x\: +2 \\
f(x) =& x^2\; +3x\; +2 \\
f(x) =& x^2\ +3x\ +2 \\   % em space
f(x) =& x^2\quad +3x\quad +2 \\ % big space
f(x) =& x^2\qquad +3x\qquad +2  % gigantic space
\end{align*}
$$

% Aslign plain text
Here is some text before by aligned text\par
\makebox[1.5cm]{S}  Isomeric state\par
\makebox[1.5cm]{Z}  Atomic number\par
\makebox[1.5cm]{A}  Atomic mass number\par
Some text after the aligned text

%% === URL ANSWERS ===

% URL , Tikz Picture Node Radius : https://tex.stackexchange.com/questions/48538/how-to-set-exact-radius-for-a-node
% URL , Tikz Picture Node Shapes : http://www.texample.net/tikz/examples/node-shapes/
% URL , Tikz Picture Define Styles : https://tex.stackexchange.com/a/47074
% URL , Tikz Picture Use Layers : https://tex.stackexchange.com/a/18201
% URL , Tikz Picture Rotate Nodes and Shapes : https://tex.stackexchange.com/a/193133
% URL , Define Variables : https://tex.stackexchange.com/questions/258/what-is-the-difference-between-let-and-def

%% === END URL ===